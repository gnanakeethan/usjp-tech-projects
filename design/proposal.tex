\documentclass[12pt,a4paper]{article}
\usepackage[paper=a4paper]{geometry}% http://ctan.org/pkg/geometry
\usepackage[english]{babel}
\usepackage[T1]{fontenc}


\usepackage{multicol}
\usepackage{lipsum}
\usepackage{setspace}
\usepackage{soul}
\usepackage{fontspec}

\usepackage{pgfgantt}
\usepackage{graphicx}
\usepackage[yyyymmdd]{datetime}
\usepackage{tikz}
\usetikzlibrary{calc}
\usepackage{indentfirst} 
\usepackage{pgfgantt}
\usepackage{rotating}
\usepackage[graphicx]{realboxes}

\usepackage{sectsty}
\usepackage{float}
%\usepackage[table,xcdraw]{xcolor}
\sectionfont{\fontsize{24pt}{36}\selectfont}
\usepackage{tocloft}
\renewcommand{\cftsecleader}{\cftdotfill{\cftdotsep}}




\setmainfont[Ligatures=TeX]{Times New Roman}
\newgeometry{top=20mm,bottom=20mm,left=40mm,right=40mm}


\begin{document}

\begin{titlepage}
\center % Center everything on the page
{\setstretch{1.2}
\includegraphics[width=2in,height=2in,keepaspectratio]{logo.png}\\[0.5cm]
\fontsize{16pt}{24}\selectfont \textbf{Project Proposal}\\[0.5cm]
\fontsize{16pt}{24}\selectfont \textbf{EGT 31303}\\[0.75cm]
\fontsize{24pt}{30}\selectfont \textbf{\uppercase{Digital Measuring System for Cloth}}\\[1.5cm]
\fontsize{16pt}{24}\selectfont \textbf{By}\\[0.5cm]
\fontsize{12pt}{12}\selectfont {
\bgroup
\def\arraystretch{2}% 
\begin{tabular}{|c|l|c|}
        \hline
        \textbf{Index No} &        \textbf{Name} &         \textbf{Marks from (5)} \\
        \hline
        INDEX NO & NAME & \\
        \hline
\end{tabular}
\egroup
}


\vspace{1.5cm}
\fontsize{12pt}{12}\selectfont \textbf {Department of Engineering Technology \\ Faculty of Technology\\University of Sri Jayewardenepura\\ Sri Lanka}\\ \today 
}


\end{titlepage}

\pagenumbering{arabic}
\setcounter{page}{1}

\section{\uppercase {Problem Definition}}

There is a pertaining problem for measuring lengths of cloth quickly and accurately. It is hard to maintain a tape straight while holding cloth as well. The existing methods such as using a Meter ruler requires more adjustments to perform perfect measurements. It is also hard to make continuous measurements of long cloths as well. \\[1cm]

{\fontsize{16pt}{24}\selectfont \textbf {Issues with Current Methods used} }\\

\begin{itemize}
  \item \textbf{Tape}
  \begin{itemize}
  	\item It is not feasible to hold both tape and cloth at the same time
	\item High precision can not be obtained easily.
	\item The tape wears over time reducing its feasibility
  \end{itemize}
  \item \textbf{Meter Rule}
  \begin{itemize}
  	\item Meter rules are quite accurate, but hard to operate.
	\item The markings on meter rule may vanish over time, making it hard to read easily. 
	\item It is not feasible for regular household usage or making small measurements.
  \end{itemize}
\end{itemize}


 \newpage
 \newgeometry{top=20mm,bottom=20mm,left=20mm,right=10mm}
\section{\uppercase {Methodology}}

We are planning to construct a connected device based on Microcontroller. It will be able to measure short and long lengths very easily. It will also be easy to measure in multiple scales at the same time. The device will also be able to connect to smart-phones and display and record readings on the smart-phone. 



\begin{itemize}
  \item \textbf{Step 1:}  Creating a rough design of the object
  \begin{itemize}
  	\item Identify proper locations for placing the parts
	\item Perfect the placement of parts for easier handling
	\item Purchase of the parts necessary for construction of the device
  \end{itemize}
  
  \item \textbf{Step 2:} Defining the wiring diagram
  \begin{itemize}
  	\item Wire and Assemble the parts with Arduino Uno 
	\item Test the proper functioning of assembled parts
	\item Make adjustments according to the issues found.
	\item \textbf{Minimum Viable Product}
  \end{itemize}
  
  \item \textbf{Step 3:} Mobile Connectivity
  \begin{itemize}
  	\item Create a mobile app(Android application) and connect it via Bluetooth
	\item Perfect the Functionality of the device to connect via Bluetooth for readings.
  \end{itemize}
  
  \item \textbf{Step 4:} Final Assembly and Testing
  \begin{itemize}
  	\item Final Testing of the device before assembling it into a unit
	\item Final Assembly
  \end{itemize}
  
\end{itemize}
 
 
 \newpage
 
\section{\uppercase {Time Schedule}} 

\begin{figure}[!htb]
\begin{ganttchart}[    vgrid,hgrid,
bar/.append style={fill=black!60},
  group/.append style={draw=black, fill=green!30},
  milestone/.append style={fill=green!100, rounded corners=3pt},
    x unit=3.5mm,
]{1}{42}
\gantttitle{Week 1}{7}
\gantttitle{Week 2}{7}
\gantttitle{Week 3}{7}
\gantttitle{Week 4}{7}
\gantttitle{Week 5}{7}
\gantttitle{Week 6}{7}\\
   \ganttgroup{Planning}{1}{11} \\
     \ganttbar{Methodology}{1}{3} \\
     \ganttbar{Cost Analysis}{4}{11} \\
   \ganttgroup{Materials}{6}{14} \\
     \ganttbar{Finalization}{6}{11} \\
     \ganttbar{Purchase}{10}{14} \\
   \ganttgroup{Develop}{8}{28} \\
     \ganttbar{Wiring Diagram}{8}{13} \\
     \ganttbar{Initial Assembly}{10}{16} \\
     \ganttbar{Programming}{12}{22} \\
     \ganttbar{Fixing Issues}{16}{28} \\
     \ganttmilestone{\textbf{MVP}}{28}\\
   \ganttgroup{Android App}{24}{38} \\
     \ganttbar{Development}{24}{38} \\
     \ganttbar{Testing}{28}{38} \\
   \ganttgroup{Device Testing}{30}{40} \\
     \ganttbar{Real Life Testing}{30}{40} \\
     \ganttbar{Fixing Issues}{32}{40} \\
     \ganttbar{Finalizing Report}{40}{42} \\
     \ganttmilestone{\textbf{Final Submission}}{42}\\
\end{ganttchart}
\end{figure}
\end{document}
