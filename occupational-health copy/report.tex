\documentclass[12pt,a4paper]{article}
\usepackage[paper=a4paper]{geometry}% http://ctan.org/pkg/geometry
\usepackage[english]{babel}
\usepackage[T1]{fontenc}


\usepackage{fancyvrb}
\usepackage{multicol}
\usepackage{lipsum}
\usepackage{setspace}
\usepackage{soul}
\usepackage{fontspec}
 \usepackage{colortbl}
\usepackage{background}
\usepackage{caption}
\usepackage{pgfgantt}
\usepackage{graphicx}
\usepackage[yyyymmdd]{datetime}
\usepackage{tikz}
\usetikzlibrary{calc}
\usepackage{indentfirst} 
\usepackage{pgfgantt}
\usepackage{rotating}
\usepackage[graphicx]{realboxes}
\usepackage{tocloft}
\usepackage{xcolor}
\usepackage{listings}

\definecolor{mGreen}{rgb}{0,0.6,0}
\definecolor{mGray}{rgb}{0.5,0.5,0.5}
\definecolor{mPurple}{rgb}{0.58,0,0.82}
\definecolor{backgroundColour}{rgb}{0.95,0.95,0.92}

\lstdefinestyle{CStyle}{
    backgroundcolor=\color{backgroundColour},   
    commentstyle=\color{mGreen},
    keywordstyle=\color{magenta},
    numberstyle=\tiny\color{mGray},
    stringstyle=\color{mPurple},
    basicstyle=\footnotesize,
    breakatwhitespace=false,         
    breaklines=true,                 
    captionpos=b,                    
    keepspaces=true,                 
    numbers=none,                    
    numbersep=5pt,                  
    showspaces=false,                
    showstringspaces=false,
    showtabs=false,                  
    tabsize=2,
    language=C
}



\newcommand{\listappendicesname}{Appendices}
\newlistof{appendices}{apc}{\listappendicesname}
\newcommand{\appendices}[1]{\addcontentsline{apc}{appendices}{#1}}

\newcommand{\newappendix}[1]{\section*{#1}\appendices{#1}}


\usepackage{sectsty}
\usepackage{float}
%\usepackage[table,xcdraw]{xcolor}
\sectionfont{\fontsize{24pt}{36}\selectfont}
\usepackage{tocloft}
\renewcommand{\cftsecleader}{\cftdotfill{\cftdotsep}}
\usepackage{afterpage}

\usepackage{listings}
\usepackage{color}
 
\definecolor{codegreen}{rgb}{0,0.6,0}
\definecolor{codegray}{rgb}{0.5,0.5,0.5}
\definecolor{codepurple}{rgb}{0.58,0,0.82}
\definecolor{backcolour}{rgb}{0.95,0.95,0.92}
 
\lstdefinestyle{mystyle}{
    backgroundcolor=\color{backcolour},   
    commentstyle=\color{codegreen},
    keywordstyle=\color{magenta},
    numberstyle=\tiny\color{codegray},
    stringstyle=\color{codepurple},
    basicstyle=\footnotesize,
    breakatwhitespace=false,         
    breaklines=true,                 
    captionpos=b,                    
    keepspaces=true,                 
    numbers=left,                    
    numbersep=5pt,                  
    showspaces=false,                
    showstringspaces=false,
    showtabs=false,                  
    tabsize=2
}
 
\lstset{style=mystyle}
\captionsetup{justification=raggedright,singlelinecheck=false}

\newcommand\blankpage{%
    \null
    \thispagestyle{empty}%
    \addtocounter{page}{-1}%
    \newpage}
\renewcommand{\baselinestretch}{1.5} 

\usetikzlibrary{calc}
\newcommand\HRule{\rule{\textwidth}{1pt}}

\setmainfont[Ligatures=TeX]{Times New Roman}
\newgeometry{top=20mm,bottom=20mm,left=28mm,right=13mm}
\usepackage[citestyle=ieee]{biblatex}
\addbibresource{report.bib}

\backgroundsetup{
color=black,
scale=1,
opacity=1,
angle=0,
contents={
\begin{tikzpicture}
\draw [line width=0.75pt] ($ (current page.north west) + (2.5cm,-1cm) $) rectangle ($ (current page.south east) + (-1cm,1cm) $);
\draw [line width=0.01pt] ($ (current page.north west) + (0cm,0cm) $) rectangle    ($ (current page.south east) + (0cm,0cm) $); 
\end{tikzpicture}
}
}

\setlength{\footskip}{20pt}


\begin{document}

\begin{titlepage}

%\begin{tikzpicture}[remember picture, overlay]
  %\draw[line width = 1pt] ($(current page.north west) + (25mm,-10mm)$) rectangle ($(current page.south east) + (-25mm,10mm)$);
%\end{tikzpicture}

\begin{flushright}
%\textbf{\uppercase{\fontsize{12}{18} \selectfont {Group No: E2}}}
\end{flushright}


\center % Center everything on the page
{\setstretch{1.5}
\includegraphics[width=2in,keepaspectratio]{logo.png}\\[0.5cm]
% \fontsize{16pt}{24}\selectfont \textbf{Project Report}\\[0.5cm]
\fontsize{24pt}{30}\selectfont \textbf{\uppercase{Assignment - 3}}\\[1.5cm]
\fontsize{16}{24}\selectfont \textbf{\uppercase{Sustainable and Renewable Energy}}\\[1.5cm]
\fontsize{16pt}{24}\selectfont \textbf{ETE 3482}\\[0.75cm]
\fontsize{16pt}{24}\selectfont \textbf{By}\\[0.5cm]
\fontsize{12pt}{12}\selectfont \textbf { Gnanakeethan Balasubramaniam \\ EGT/16/437}\\[0.5cm]

\vspace*{\fill}
\fontsize{12pt}{12}\selectfont \textbf {Department of Engineering Technology \\ Faculty of Technology\\University of Sri Jayewardenepura\\ Sri Lanka \\ }
\vspace*{\fill}

\today
}

%\vfill % Fill the rest of the page with whitespace


\end{titlepage}


\pagenumbering{roman}
\renewcommand{\baselinestretch}{1}\normalsize
\tableofcontents
\renewcommand{\baselinestretch}{1.5}\normalsize

\listoffigures
 
\listoftables

\listofappendices

\newpage

\pagenumbering{arabic}
%\pagenumbering{gobble}
\setcounter{page}{1}
%\addcontentsline{toc}{section}{Unnumbered Section}
\section{Biomass Combustion Systems and Boilers}

\subsection{Combustion Systems}

\subsubsection{Stationary Grate}
These type of grate were used in 1800s. They are fairly easy to build, but require constant attention. These rely on gravity to move the fuel. Usually they require a inclination of 30-50 degrees.  \cite[p.~205]{steam-generation}


\subsubsection{Mechanical Grate}

These have large chains that move the biomass into the burner. Thus they also feature ash removal. These are also inclined. They operate at varying speeds according to industrial needs
\cite[p.~206]{steam-generation}

\subsection{Boilers}

\subsubsection{Large Volume Boilers}
These are usually fire tube or gas tube boilers. These are limited by steam capacity and operating pressure because of their design. The design has not changed since Scottish marine boilers of 1800s.


In these boilers, the water circulates downwards at the edges of the boilers. 
\cite[p.~89]{steam-generation}


\subsubsection{Natural Circulation Boiler}

The natural circulation is based on density differences. It is caused by the density difference between saturated water and heater water partially filled with steam bubbles. Here the tubes are connected in a loop. 
\cite[p.~89]{steam-generation}

\subsubsection{Assisted Circulation}
Assisted Circulation is usually employed in heat recovery steam generator boilers and high pressure units. A synonym for assisted circulation is forced circulation.
\cite[p.~89]{steam-generation}

\subsubsection{Once Through Boiler}

In this process the water flows continuously through the boiler, coming out as 100\% of steam at the main steam outlet. 

The feedwater purity must be high as any kind of contamination would cause serious damage to the boiler. 

\cite[p.~89]{steam-generation}

\section{Advantages of Pyrolysis}

\begin{itemize}
\item It can be used to process a wide variety of feedstocks
\item Reduces the wastes going to landfills
\item Reduces the risk of water contamination
\item Construction of a Pyrolysis plant can be faster than other plants.
\end{itemize}


\section{Gastrification and Pyrolysis}

Pyrolysis is thermal decomposition of volatile components in a organic substance in the absence of oxygen and occurs in range of 200-760 degree celcius whereas gastrification occurs in the temperature range of 480-1650 degree celcius, in addition to the thermal decomposition, the non-volatile carbon char is converted to additional syngas. 

\section{Environmental Benefit of Anaerobic Digestion}
\begin{itemize}
        \item The organic materials are not ending up in landfills and recycled into the ecosystem.
        \item The Anaerobic digestion results in production of Methane, which is a useful energy source if properly filtered.
        \item The remnants of Anaerobic digestion is a nutrient rich slurry which can be added to soil to enrich it with essential nutrients.
        \item Methane emissions from the land-fills will be significantly reduced as organic materials are removed from dumps. 

\end{itemize} 

\section{Chemical Process of Anaerobic Digestion}

$C_{6}H_{12}O_{6}$ $\rightarrow$ $3CO_{2}$ + $3CH_{4}$

\section{Ethanol Production using Sugar cane}

\begin{itemize}
        \item The sugar cane is obtained and cleaned of sand,dirt, and metals.
            \item Sugar bagasse is removed
            \item Juice is treated and filtered for sand, fibers and impurities
            \item Juice is then concentrated
            \item Then it is sterilized
            \item Fermentation Process Starts
            \item Gas Absorption from Fermentation
            \item Distillation and Rectification produces 2nd Grade ethanol
            \item Further dehydration produces Anhydrous bioethanol
\end{itemize}
Ref: Chemical engineering research and design(2009) p.1209

\newpage
%\section*{\textbf{References}}

\printbibliography[heading=bibliography,title={References}]

\end{document}
